% Options for packages loaded elsewhere
\PassOptionsToPackage{unicode}{hyperref}
\PassOptionsToPackage{hyphens}{url}
%
\documentclass[
]{article}
\usepackage{amsmath,amssymb}
\usepackage{lmodern}
\usepackage{ifxetex,ifluatex}
\ifnum 0\ifxetex 1\fi\ifluatex 1\fi=0 % if pdftex
  \usepackage[T1]{fontenc}
  \usepackage[utf8]{inputenc}
  \usepackage{textcomp} % provide euro and other symbols
\else % if luatex or xetex
  \usepackage{unicode-math}
  \defaultfontfeatures{Scale=MatchLowercase}
  \defaultfontfeatures[\rmfamily]{Ligatures=TeX,Scale=1}
\fi
% Use upquote if available, for straight quotes in verbatim environments
\IfFileExists{upquote.sty}{\usepackage{upquote}}{}
\IfFileExists{microtype.sty}{% use microtype if available
  \usepackage[]{microtype}
  \UseMicrotypeSet[protrusion]{basicmath} % disable protrusion for tt fonts
}{}
\makeatletter
\@ifundefined{KOMAClassName}{% if non-KOMA class
  \IfFileExists{parskip.sty}{%
    \usepackage{parskip}
  }{% else
    \setlength{\parindent}{0pt}
    \setlength{\parskip}{6pt plus 2pt minus 1pt}}
}{% if KOMA class
  \KOMAoptions{parskip=half}}
\makeatother
\usepackage{xcolor}
\IfFileExists{xurl.sty}{\usepackage{xurl}}{} % add URL line breaks if available
\IfFileExists{bookmark.sty}{\usepackage{bookmark}}{\usepackage{hyperref}}
\hypersetup{
  pdftitle={Spørgeskema om kunstig intelligens},
  hidelinks,
  pdfcreator={LaTeX via pandoc}}
\urlstyle{same} % disable monospaced font for URLs
\usepackage[margin=1in]{geometry}
\usepackage{graphicx}
\makeatletter
\def\maxwidth{\ifdim\Gin@nat@width>\linewidth\linewidth\else\Gin@nat@width\fi}
\def\maxheight{\ifdim\Gin@nat@height>\textheight\textheight\else\Gin@nat@height\fi}
\makeatother
% Scale images if necessary, so that they will not overflow the page
% margins by default, and it is still possible to overwrite the defaults
% using explicit options in \includegraphics[width, height, ...]{}
\setkeys{Gin}{width=\maxwidth,height=\maxheight,keepaspectratio}
% Set default figure placement to htbp
\makeatletter
\def\fps@figure{htbp}
\makeatother
\setlength{\emergencystretch}{3em} % prevent overfull lines
\providecommand{\tightlist}{%
  \setlength{\itemsep}{0pt}\setlength{\parskip}{0pt}}
\setcounter{secnumdepth}{-\maxdimen} % remove section numbering
\ifluatex
  \usepackage{selnolig}  % disable illegal ligatures
\fi

\title{Spørgeskema om kunstig intelligens}
\author{}
\date{\vspace{-2.5em}}

\begin{document}
\maketitle

\hypertarget{spuxf8rgeskema-om-brug-af-digitale-teknologier-og-kunstig-intelligens-i-sundhedssystemet}{%
\section{Spørgeskema om brug af digitale teknologier og kunstig
intelligens i
sundhedssystemet}\label{spuxf8rgeskema-om-brug-af-digitale-teknologier-og-kunstig-intelligens-i-sundhedssystemet}}

Digital teknologi og kunstig intelligens har fået en større plads i
vores daglige liv og i sundhedsvæsnet. Vi ser en større udvikling af
forskellige teknologier, som kan ændre både arbejdsgange for
sundhedspersonalet og måder patienterne får støttet og behandling.
Eksempler på disse teknologier: online konsultationsplatforme,
robotkirurgi, mobilapps til monitorering af sundhed.

Vi vil gerne høre din holdning til udviklingen af disse nye teknologier
i sundhedssystemet.

\textbf{Q1 Hvordan bruger du generelt internettet?}

\begin{itemize}
\item[$\square$]
  Mest på min hjemmecomputer (stationær, bærbar)
\item[$\square$]
  Mest på mobilenheder (smartphone, tablet)
\item[$\square$]
  Jeg bruger ikke internettet
\end{itemize}

\textbf{Q2 Hvor ofte burger du internettet?}

\emph{Vis kun dette spørgsmål til dem som har svaret `Mest, \ldots{}' i}
\_ \textbf{Q1} \_

\textbf{Q3 Har du nogensinde brugt følgende sundhedsrelaterede
teknologier}  \textbf{?}

\emph{Matrix: Yes/No}

Sundhedsrelateret apps på smartphone

\begin{itemize}
\tightlist
\item[$\square$]
  Ja
\item[$\square$]
  Nej
\end{itemize}

Kropsbårne apparater til måling af helbred (f.eks. skridt tæller, smart
ur)

\begin{itemize}
\tightlist
\item[$\square$]
  Ja
\item[$\square$]
  Nej
\end{itemize}

Kropsbårne apparater ordineret af læge (f.eks. høre apparater)

\begin{itemize}
\tightlist
\item[$\square$]
  Ja
\item[$\square$]
  Nej
\end{itemize}

Online sundhedsservice (f.eks. online konsultationer)

\begin{itemize}
\tightlist
\item[$\square$]
  Ja
\item[$\square$]
  Nej \emph{(følgende svar er kun for dem med diabetes)}
\end{itemize}

Blodsukker sensorer (f.eks. flash monitoring eller stik i fingeren)

\begin{itemize}
\tightlist
\item[$\square$]
  Ja
\item[$\square$]
  Nej
\end{itemize}

Kontinuerlig blodsukker måling (f.eks. Libre Pro sensorer)

\begin{itemize}
\tightlist
\item[$\square$]
  Ja
\item[$\square$]
  Nej
\end{itemize}

Insulinpumpe

\begin{itemize}
\tightlist
\item[$\square$]
  Ja
\item[$\square$]
  Nej
\end{itemize}

Insulinpen

\begin{itemize}
\tightlist
\item[$\square$]
  Ja
\item[$\square$]
  Nej
\end{itemize}

DIY looping system

\begin{itemize}
\tightlist
\item[$\square$]
  Ja
\item[$\square$]
  Nej
\end{itemize}

\textbf{Q4 I hvilket omfang ville du støtte delingen af dine
anonymiserede sundhedsdata til sundhedsvidenskabelig forskning hos
følgende organisationer?}

\emph{Matrix: 1=} Meget sandsynligt\_, 2=Mindre sandsynligt, 3=Ikke
sikker, 4= Mindre usandsynligt, 5= Mindre usandsynligt\_

Private virksomheder (f.eks. en sundhedsteknologiske virksomhed): 1-5 -
{[} {]} Meget sandsynligt

\begin{itemize}
\item[$\square$]
  Mindre sandsynligt
\item[$\square$]
  Ikke sikker
\item[$\square$]
  Mindre usandsynligt
\item[$\square$]
  Mindre usandsynligt
\end{itemize}

Offentlige organisationer (f.eks. universiteter, hospitaler): 1-5

\begin{itemize}
\item[$\square$]
  Meget sandsynligt
\item[$\square$]
  Mindre sandsynligt
\item[$\square$]
  Ikke sikker
\item[$\square$]
  Mindre usandsynligt
\item[$\square$]
  Mindre usandsynligt
\end{itemize}

Offentlig-Privat samarbejde (f.eks. samarbejde mellem hospitaler og
virksomheder): 1-5

\begin{itemize}
\item[$\square$]
  Meget sandsynligt
\item[$\square$]
  Mindre sandsynligt
\item[$\square$]
  Ikke sikker
\item[$\square$]
  Mindre usandsynligt
\item[$\square$]
  Mindre usandsynligt
\end{itemize}

\textbf{Q5 Ville du støtte at Aarhus Universitet laver en stor database
med rutinemæssigt indsamlede, anonyme sundhedsdata i Region Midt til
forskning af digital teknologier?}

\begin{itemize}
\tightlist
\item[$\square$]
  Ja
\item[$\square$]
  Nej
\item[$\square$]
  Ved ikke
\end{itemize}

\textbf{Q6 Ville du acceptere, at dine anonymiserede data også ville
blive inkluderet i sådan en database?}

\begin{itemize}
\tightlist
\item[$\square$]
  Ja
\item[$\square$]
  Nej
\item[$\square$]
  Ved ikke
\end{itemize}

\textbf{Q7 Har du hørt om kunstig intelligens tidligere?}

\begin{itemize}
\tightlist
\item[$\square$]
  Ja, jeg har hørt om det og jeg har en ide om hvad det er
\item[$\square$]
  Ja, jeg har hørt om det, men jeg ved ikke hvad det er
\item[$\square$]
  Nej
\end{itemize}

\textbf{Q8 I hvilken grad tror du at kunstig intelligens kan have en
gavnlig betydning på følgende områder?}

\emph{Vis kun dette spørgsmål til dem som har svaret `Ja, \ldots{}' i}
\_ \textbf{Q7} \_

\emph{Matrix: 1=Stor betydning, 2= Overvejende betydning, 3=Lille
betydning, 4=Ingen betydning, 5= Ved ikke}

Forbedrer adgang til behandling: 1-5

Forbedrer Opfølgning af patienter: 1-5

Forbedrer effektivitet i sundhedspersonalets arbejde: 1-5

Forbedrer kommunikation under behandlingsforløb: 1-5

Forbedrer observering og forebyggelse i sundhed: 1-5

Forbedrer sikkerhed i behandling: 1-5

Optimerer og reducerer behandlingsbyrden: 1-5

Andet: \_\_\_\_\_\_\_\_\_\_\_\_\_ (\emph{skal måske indføres som et
separate spørgsmål})

\textbf{Q9 I hvilket grad er du bekymret for følgende risici ved brug af
kunstig intelligens i sundhedssystemet?}

\emph{Vis kun dette spørgsmål til dem som har svaret `Ja, \ldots{}' i}
\_ \textbf{Q7} \_

\emph{Matrix: 1=Meget bekymret, 2=Overvejende bekymret, 3= Lidt
bekymret, 4=Ikke bekymret, 5=Ved ikke}

Øger risiko for hacking og datamisbrug: 1-5

Forstyrrer patienternes liv (f.eks. observering og alarmer): 1-5

Bliver sværere at forstå, hvordan beslutninger bliver taget: 1-5

Bliver sværere at udfordre råd eller søge anden vurdering: 1-5

Erstatter behandling fra sundhedspersonalet: 1-5

Forringelse af forholdet mellem patienten og sundhedspersonale: 1-5

Vil måske ikke være lige tilgængeligt for alle: 1-5

Andet:\_\_\_\_\_\_\_\_\_\_\_\_\_ (\emph{skal måske indføres som et
separate spørgsmål})

\textbf{10) Tror du at fordele ved brug af kunstig intelligens i
sundhedssystemet opvejer risici?}

\emph{Vis kun dette spørgsmål til dem som har svaret `Ja, \ldots{}' i}
\_ \textbf{Q7} \_

\begin{itemize}
\tightlist
\item[$\square$]
  Fordele overstiger risici
\item[$\square$]
  Risici og fordele er lige store
\item[$\square$]
  Risici overstiger fordele
\item[$\square$]
  Ved ikke
\end{itemize}

\textbf{Scenarier}

I følgende vil vi præsentere nogle situationer, hvor ny teknologi,
kropbårne apparater, eller kunstig intelligens kunne supplere eller
erstatte sundhedspersonalets ekspertise. Tænk hver situation kort
igennem og angiv, om du ville være villig til at benytte dig af disse
teknologier.

\textbf{S1} Hvis der var solid viden om at automatisk billedanalyser af
din hud udført via kunstig intelligens var lige så god eller bedre til
at finde modersmærker med risiko for udvikling af hudkræft sammenlignet
med undersøgelse af en dermatolog. Ville du acceptere brug af denne
teknologi til din screening og diagnose?

\begin{itemize}
\tightlist
\item[$\square$]
  Nej, det ville jeg ikke
\item[$\square$]
  Ja, men kun med kontrol af sundhedspersonale
\item[$\square$]
  Ja, det kunne erstatte sundhedspersonale
\item[$\square$]
  Ved ikke
\end{itemize}

\textbf{S2} Hvis der var solid viden om at brug af kropsbårne apparater
(smartphone, smartwatch, mm.) til at måle patienter derhjemme, virkede
lige så godt eller bedre til at forudsige forværringer af kronisk sygdom
sammenlignet med observationer hos lægen. Ville du acceptere at bruge
disse apparater?

\begin{itemize}
\tightlist
\item[$\square$]
  Nej, det ville jeg ikke
\item[$\square$]
  Ja, men kun med kontrol af sundhedspersonale
\item[$\square$]
  Ja, det kunne erstatte sundhedspersonale
\item[$\square$]
  Ved ikke
\end{itemize}

\textbf{S3} Hvis der var solid viden om at et kunstig intelligens
program (automatisk samtaleapplikation eller ``chatbot'') som svarer på
opkald til alarmcentralen virkede lige så godt eller bedre til at
bedømme alvoren af patientens henvendelse, sammenlignet med den
nuværende service hos alarmcentralen. Ville du acceptere brugen af denne
teknologi til din behandling?

\begin{itemize}
\tightlist
\item[$\square$]
  Nej, det ville jeg ikke
\item[$\square$]
  Ja, men kun med kontrol af sundhedspersonale
\item[$\square$]
  Ja, det kunne erstatte sundhedspersonale
\item[$\square$]
  Ved ikke
\end{itemize}

\textbf{S4} Hvis der var solid viden om at målinger af bevægelse gennem
smart tøj virkede lige så godt eller bedre i tilpasningen af patientens
rehabilitering sammenlignet opfølgning hos en fysioterapeut.

Ville du acceptere at denne teknologi blev anvendt til din behandling?

\begin{itemize}
\tightlist
\item[$\square$]
  Nej, det ville jeg ikke
\item[$\square$]
  Ja, men kun med kontrol af sundhedspersonale
\item[$\square$]
  Ja, det kunne erstatte sundhedspersonale
\item[$\square$]
  Ved ikke
\end{itemize}

\_\textbf{(}\_\_\textbf{Vis kun scenarie 5 og 6 til dem som har
selvrapporteret diabetes)}\_

\textbf{S5} Hvis der var solid viden om at automatisk billedanalyser af
dine øjne gennem kunstig intelligens var lige så godt eller bedre til at
finde diabetisk retinopati sammenlignet med øjenundersøgelse af en
optometrist el. øjenlæge. Ville du acceptere brugen af denne teknologi
til din screening og diagnose?

\begin{itemize}
\tightlist
\item[$\square$]
  Nej, det ville jeg ikke
\item[$\square$]
  Ja, men kun med kontrol af sundhedspersonale
\item[$\square$]
  Ja, det kunne erstatte sundhedspersonale
\item[$\square$]
  Ved ikke
\end{itemize}

\textbf{S6} Hvis der var solid viden om at automatisk analyse af dit
blodsukker gennem kunstig intelligens virkede lige så godt eller bedre
til at tilpasse din medicinske behandling sammenlignet med lægens
vurdering. Ville du acceptere at bruge denne teknologi til udskrivelse
og justering af medicin?

\begin{itemize}
\tightlist
\item[$\square$]
  Nej, det ville jeg ikke
\item[$\square$]
  Ja, men kun med kontrol af sundhedspersonale
\item[$\square$]
  Ja, det kunne erstatte sundhedspersonale
\item[$\square$]
  Ved ikke
\end{itemize}

\end{document}
